\chapter{Key Parameters}
\label{ch:key-params}

In ubiquitous computing of Smart Home networks, the different wireless networking protocols do not perform well in any Smart Home scenario. Therefore, the wireless networking technology which performs best in the specific Smart Home application should be choosen. To perform a comparison of wireless networking technologies in accordance to different Smart Home applications, a wireless protocol should address the following issues: \footcite[Cf.][7289--7293]{Mendes2015}

\begin{itemize}
  \item Energy efficiency: The low power consumption of the transmission layers.
  \item Bandwidth efficiency: Spread spectrum efficiency, channel width, and coexistence stability.
  \item Scalability: The number of nodes supported in a network.
  \item Transmission Range: The theoretical coverage range and the penetration of obstacles.
  \item Throughput: The maximum rate of transmission.
  \item Latency: Theoretical latency between two nodes.
  \item Mobility: Efficient mobile node management.
\end{itemize}

The taxonomy of parameters for the wireless networking technologies is shown in \cref{fig:taxonomy_of_parameters}. The evaluation of the parameters is discussed in \cref{ch:eval}.

\begin{figure}[ht]
  \centering
  \includegraphics[
    width=12cm,
  ]{images/taxonomy_of_parameters}
  \caption{Taxonomy of parameters for evaluation of wireless networking technologies, adapted from: \cite[7290]{Mendes2015}}
  \label{fig:taxonomy_of_parameters}
\end{figure}

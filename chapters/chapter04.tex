\chapter{Wireless Personal Area Networking Technologies}
\label{ch:wpan}

In this chapter the relevant wireless networking technologies, which are used in Smart Home networks, are discussed. Referring to the previous chapter \glspl{whn} consists of the applications spanning from body near sensor communication to multi-room communication. Therefore this chapter examines the following wireless networking technologies: \footcite[Cf.][]{Hunn2010}

\begin{itemize}
  \item Bluetooth
  \item Bluetooth Smart
  \item \gls{uwb}
  \item ZigBee
  \item \gls{6lowpan}
  \item Wi-Fi
\end{itemize}

For a proper comparison of those wireless networking technologies in context with this thesis, they need to be discussed in more detail to clarify the key mechanisms behind each technology and how the data transport is designed.

\section{Bluetooth}
\label{sec:bluetooth}

Bluetooth was originally designed for replacement of cables in \gls{wpan}. The IEEE 802.15 group was founded to develop an open standard for the first \gls{wpan} technology. At the time of writing, IEEE 802.15.1 Bluetooth is part and parcel of every mobile phone currently manufactured on this planet and also known under the synonym Bluetooth Classic. Because the IEEE 802.15.1 standard is operating at 2.45 GHz where the IEEE 802.11 standard operates, coexistence and interference are part of the standard. \footcite[Cf.][19]{Al-Qutayri2010}

The frequency band used by those standards is the \gls{ism}. It ranges from 2.4 GHz to 2.4835 GHz. \footcite[Cf.][19]{Al-Qutayri2010}

The data rate of Bluetooth is considerably lower than those of IEEE 802.11 Wi-Fi networks, but Bluetooth incorporates a mechanism featuring voice applications. \footcite[Cf.][445-446]{Pahlavan2009}

The Bluetooth standard defines three operating scenarios. The first scenario is cable replacement of \gls{wpan} devices. This enables the use of Bluetooth enabled mice or keyboards. The second scenario defines Bluetooth as an ad-hoc networking technology for interconnecting several users, for example during meetings and conferences. The third scenario enables Bluetooth devices to operate an access point to the voice and data services of cellular, wired or satellite networks. \footcite[Cf.][445-446]{Pahlavan2009}

\subsection{Networking Architecture}

\Cref{fig:piconet_scatternet} shows the ad-hoc networking architecture of Bluetooth nodes. The Bluetooth standard defines piconets identifying four states: master (M), slave (S), standby (SB) and park (P). Whereas M and S nodes are active, SB nodes are in a waiting position and P nodes are inactive. Nodes in P state have released their \gls{mac} address, all other node states have a registered \gls{mac} address. A piconet can contain up to 200 active nodes. Slave nodes have the ability to join multiple piconets simultaneously. Such an interconnection between multiple piconets is called scatternet. \footcite[Cf.][446-448]{Pahlavan2009}

\begin{figure}[ht]
  \centering
  \includegraphics[
    width=8cm,
  ]{images/piconet_scatternet}
  \caption{Relationship of Bluetooth piconet and scatternet, adapted from: \cite[690]{Garg2007}}
  \label{fig:piconet_scatternet}
\end{figure}

\subsection{Bluetooth Protocol Stack}

Application developers benefit from the protocol stack, which is defined by the IEEE 802.15.1 standard. It defines two layers, the transport protocol and the middleware protocol. As shown in \cref{fig:bt_protocol_stack}, the transport protocol defines the radio, baseband, link manager and the audio components and the middleware protocol encompasses the \gls{l2cap}, \gls{rfcomm}, \gls{tcs} and \gls{sdp}. The \gls{hci} is the bridge between the hardware and the software components. From bottom up, the radio layer defines the modem used for data and voice transmission. The baseband layer provides low-level link control, by specifying packet coding and encryption. The next layer is the link manager responsible for configuring links to other Bluetooth devices. The link manager provides authentication, piconet statistics and general link monitoring and control functions, including power state information and traffic scheduling. The audio data processing layer is the last layer of the transport protocol group. The middleware protocol group is set upon the \gls{hci} and provides a basic layer for data transmission to upper layer protocols, the \gls{l2cap}. This layer can process data packets of the maximum size of 64 kbytes. The top layers of the Bluetooth protocol stack are application specific. The \gls{sdp} is used to locate services within a piconet such as printing services. The \gls{tcs} layer is responsible for establishing and monitoring telephone calls. The \gls{rfcomm} is the layer providing cable replacement functionality to applications. \footcite[Cf.][448]{Pahlavan2009}

\begin{figure}[ht]
  \centering
  \includegraphics[
    width=7cm,
  ]{images/bt_protocol_stack}
  \caption{The Bluetooth protocol stack, adapted from: \cite[448]{Pahlavan2009}}
  \label{fig:bt_protocol_stack}
\end{figure}

\subsection{Physical Layer}

To establish a physical connection, Bluetooth uses a \gls{fhss} modem. The antenna has a nominal power of 0 dBm, which equals a coverage radius of 10 meters. Optionally, the antenna can operate at 20 dBm power to increase the coverage to 100 meters. The \gls{fhss} defines a transmission mechanism of radio signals by hopping the carriar signals frequency between a defined frequency band using a pseudorandom algorithm. Bluetooth uses 79 defined frequency channels within the frequency band from 2.402 GHz and 2.480 GHz. The channel hopping rate is 1600 Hz. The radio signals are modulated using the \gls{gfsk} algorithm. The \gls{gfsk} is a \gls{fsk} modulation whereas the baseband signal is smoothed by gaussian filter stage. \Cref{fig:gfsk_modulation} illustrates the \gls{gfsk} modulation algorithm using four different symbols 00, 01, 10 and 11. Each symbol corresponds to a different frequency. Bluetooth uses a \gls{gfsk} algorithm which represents a binary zero at negative frequency deviation and a binary one at positive frequency deviation. The transmission rate of \gls{gfsk} modulated data is 1 MHz. \footcite[Cf.][450-451]{Pahlavan2009}

\begin{figure}[ht]
  \centering
  \includegraphics[
    width=6cm,
  ]{images/gfsk_modulation}
  \caption{The \gls{gfsk} modulation, source: \cite{Gast2005}}
  \label{fig:gfsk_modulation}
\end{figure}

The 79 frequency channels are separated into even and odd segments. Each segment covers 32 frequency channels. The initial sequence starts at one of the 79 frequency channels of the spectrum and consists of two 32 hops segments, which correlates to 64 MHz. After each segments sequence completed, the sequence is varied and shifted by 16 channels. This mechanism ensures a uniform probability at which each frequency channel is used. \footcite[Cf.][451]{Pahlavan2009}

\subsection{Medium Access Control Mechanism}

Bluetooth uses a fast frequency hopping \gls{cdma} method using a link polling mechanism. A packet transmission time slot is rather short (625 \textmu s) providing stable performance although in interference situations. Each piconet uses a different spreading sequence and due to this fact, the coexistence of ten piconets are allowed. \footcite[Cf.][452]{Pahlavan2009}

\subsection{Packet Frame Formats}

A one-slot packet (625 \textmu s) is the basic Bluetooth packet. This packet can be extended to three and five slots. An M node can therefore poll multiple S nodes simultaneously. This ability makes it possible to form piconets. \footcite[Cf.][452-453]{Pahlavan2009}

\begin{figure}[ht]
  \centering
  \includegraphics[
    width=14.5cm,
  ]{images/bt_packet_format}
  \caption{Overall Bluetooth packet frame format, source: \cite[453]{Pahlavan2009}}
  \label{fig:bt_packet_format}
\end{figure}

\Cref{fig:bt_packet_format} shows the overall packet format and structure. The frame contains the following three fields: \footcite[Cf.][453-454]{Pahlavan2009}

\begin{itemize}
  \item Access code: This field has a length of 72 bits.
  \item Header: The header field length is 54 bits.
  \item Payload: The payload length depends on the type of payload transmitted and is up to 2744 bits long, which is equivalent to five slots.
\end{itemize}

The access code field contains a 4 bit preamble, a 64 bit synchronization parameter property used for correlation and a 4 bit trailer. The unique \gls{mac} address is used as the seed to derive the frequency hopping sequence out of the synchronization parameter. The IEEE 802.15.1 defines four different types of access codes. The type to identify the M node for the piconet is unique. The S nodes use the second type and standby nodes in the inquiring phase have a different type. The fourth type is used to identify special device nodes, such as printers or mobile phones. The header field contains three 18 bits wide codes. At the beginning of each code is a 3 bit S address identifier followed by a 4 bits packet type, 3 bits representing different status codes and 8 bits for error detection and correction. The 3 bits wide S address identifier allows the existence of seven active S nodes. The payload type is specified through the 4 bit packet type identifier. Therefore 16 different payload types can be identified. The IEEE 802.15.1 standard defines six asynchronous connectionless and three synchronous connection oriented payload types. The asynchronous payload types are usually used for data communication and the synchronous payload types primarily for voice communication. The standard also defines a payload type for combined data and voice communication, as well as four link control payload types. \footcite[Cf.][453-454]{Pahlavan2009}

\section{Bluetooth Smart}
\label{sec:bluetooth_smart}

Bluetooth Smart is the marketing name for \gls{ble}, introduced in 2010 with the Bluetooth standards version 4.0. This specification includes the classic Bluetooth standard (see \cref{sec:bluetooth}) and additionally a low energy Bluetooth standard. Bluetooth Classic is backwards compatible with the prior versions of the standard. Bluetooth Smart is a totally different standard and therefore Bluetooth Classic devices implementing a standards version prior to 4.0 are not compatible with the newly introduced standard. Although, Bluetooth Smart is even not compatible with the Bluetooth Classic 4.0 specification in any configuration. \Cref{table:bt_configurations} illustrates the different specification configurations of the Bluetooth standard. \footcite[Cf.][3]{Townsend2014}

\begin{table}[h!]
  \renewcommand{\arraystretch}{2}
  \newcolumntype{P}[1]{>{\raggedright}p{#1}}
  \centering
  \begin{tabular}{ P{4.5cm} P{4.5cm} >{\arraybackslash}P{4.5cm} }
    \toprule
    \rowcolor{Gainsboro}
    \textbf{Device} & \textbf{Bluetooth Classic support} & \textbf{Bluetooth Smart support} \\
    \midrule
    \textbf{Pre-4.0~Bluetooth} & Yes & No \\
    \textbf{4.x~Single-Mode \\ Bluetooth Smart} & No & Yes \\
    \textbf{4.x~Dual-Mode \\ Bluetooth Smart Ready} & Yes & Yes \\
    \bottomrule
  \end{tabular}
  \caption{Overview of the Bluetooth specification configurations, source: \cite[4]{Townsend2014}}
  \label{table:bt_configurations}
  \vspace{-1.5em}
\end{table}

In contrast to the Bluetooth Classic standard, Bluetooth Smart can handle much larget piconet sizes. Bluetooth Smart uses a 32 bit wide access address and therefore can address thousands of devices. Depending on the connection interval the maximum number of supported slaves for a master is 11 for 7.5 ms connection interval up to a maximum of 5917 slaves at 4 s connection interval. \footcite[Cf.][11746-11747]{Gomez2012a}

\subsection{Networking Architecture}

The network topology is quite different from those of Bluetooth Classic. Bluetooth Smart devices can operate in three different topologies: point-to-point, point-to-multipoint and star. Bluetooth Smart devices which are communicating in point-to-point or star topology are forming a piconet. \footcite[Cf.][77-78]{Gratton2013}

The Bluetooth Smart piconet identifies three states: The master (M), slave (S) and parking (P). A Bluetooth Smart piconet S node can connect to only one M node. \footcite[Cf.][77-78]{Gratton2013}

The point-to-multipoint topology in Bluetooth Smart identifies the broadcasting communication mechanism specified in the Bluetooth Smart standard. \footcite[Cf.][9]{Townsend2014}

\subsection{Bluetooth Smart Protocol Stack}

\begin{figure}[ht]
  \centering
  \includegraphics[
    width=10cm,
  ]{images/bt_smart_protocol_stack}
  \caption{The Bluetooth Smart protocol stack, source: \cite[16]{Townsend2014}}
  \label{fig:bt_smart_protocol_stack}
\end{figure}

\Cref{fig:bt_smart_protocol_stack} shows the structure of the Bluetooth Smart protocol stack into three blocks: The Controller, Host and Application blocks. Each block is subdivided into several layers. \footcite[Cf.][15]{Townsend2014}

From the bottom up, the first is the Controller block which contains the physical and the link layer. The physical layer is responsible for modulation and demodulation of the analog signals from the antenna, and transforming those analog signals into digital output signals for the upper layers. The link layer contains hardware and software components for controlling the physical layer and a basic encryption mechanism.  \footcite[Cf.][16-18]{Townsend2014}

The \gls{hci} is the bridge between the Controller and Host block. It defines a standard protocol for the communication between the host and a controller. This protcol includes several commands and events for controlling, monitoring and state propagation. \footcite[Cf.][24-25]{Townsend2014}

The next block in the hierarchy of the Bluetooth Smart protocol stack is the Host. The \gls{l2cap} layer defines the foundation for the upper layers and is responsible for protocol multiplexing and packet fragmentation. The \gls{l2cap} of the Bluetooth Smart standard is able to handle two protocols: The \gls{att} for data transmission and the \gls{smp} for security key exchange. Each protocol has attached a profile: The \gls{gatt} and the \gls{gap}. \footcite[Cf.][25-33]{Townsend2014}

The Application block builds the last layer of the Bluetooth Smart protocol stack.

\subsection{Physical Layer}

The physical layer of the Bluetooth Smart is nearly identical to the physical layer of Bluetooth Classic. It uses a frequency band from 2.4 GHz to 2.4835 GHz subdivided into 40 frequency channels. The last three channels are used for advertising, broadcasting and connection establishing. The other 37 channels are used for data transmission. \footcite[Cf.][16-17]{Townsend2014}

\subsection{Packet Frame Formats}

\begin{figure}[ht]
  \centering
  \includegraphics[
    width=11.5cm,
  ]{images/bt_smart_packet_format}
  \caption{The Bluetooth Smart packet frame format, source: \cite[174]{Minoli2013}}
  \label{fig:bt_smart_packet_format}
\end{figure}

The packet format of Bluetooth Smart has a total length of 47 bytes. As shown in \cref{fig:bt_smart_packet_format}, the packet is structured into a 1 byte preamble, a 4 byte access address, a \gls{pdu} of 39 bytes length and a 3 byte \gls{crc}. The \gls{pdu} consists of the 1 byte advertising header, the 1 byte payload length, 6 bytes of advertiser address and 31 bytes payload. \footcite[Cf.][174]{Minoli2013}

\section{UWB}
\label{sec:uwb}

The IEEE 802.15 working group formed a team to further standardize \gls{wpan} technologies. The IEEE 802.15.3 defines a standard for \gls{uwb} technologies. \gls{uwb} is different from \gls{nb} or spread spectrum technologies, due its use of a ultra wide band of spectrum. In general, \gls{uwb} is defined to any radio frequency transmission system using more than 20 percent of their baseband frequency, or more than 500 MHz. \footcite[Cf.][145-146]{Lau2009}

\gls{uwb} signal generation is commonly done by using either \gls{single_uwb} or \gls{multi_uwb}. In \gls{single_uwb}, signals are transmitted using very short pulses between 10 to 1000 picoseconds occupying an extreme wide frequency band of several hundreds of MHz. The \gls{multi_uwb} structure the complete bandwidth into smaller non-overlapping frequency groups and transmit them simultaneously. \footcite[Cf.][146]{Lau2009}

\subsection{Direct Sequence Ultra Wideband}

The \gls{single_uwb} uses technology from the cellular networking, the \gls{dsss}. In \gls{single_uwb}, the whole bandwidth from 3.1 GHz to 10.6 GHz is divided into two bands, a smaller low band from 3.1 - 4.9 GHz and a wider high band from 6.2 - 9.7 GHz. The frequency gap between the low and high band is not taken into account to minimize coexistence issues with the 5 GHz band used by IEEE 802.11 technologies. \footcite[Cf.][470-471]{Pahlavan2009}

\subsection{Multiband Orthogonal Frequency-Division Multiplexing}

In the style of the \gls{single_uwb}, the \gls{multi_uwb} assimilates an existing technology. \gls{multi_uwb} uses the \gls{ofdm} mechanism widely used in IEEE 802.11 technologies. Therefore \gls{multi_uwb} divides the available bandwidth ranging from 3.1 GHz to 10.6 GHz into 15 bands. Each band has a bandwidth of 528 MHz. \footcite[Cf.][474]{Pahlavan2009}

\subsection{Summary of UWB technologies}

\gls{uwb} signals have less interference and multipath issues than other \gls{nb} technologies, due to their extreme wide bandwidth. They allow transmission of high bandwidth media streams, although the transmission is energy efficient and has a very small energy footprint. \footcite[Cf.][146]{Lau2009}

\section{IEEE 802.15.4}
\label{sec:}

IEEE 802.15.4 defines the physical and the link layers for \glspl{wpan}. IEEE 802.15.4 operates on three different bands using \gls{dsss}, whereas every band defines a different bandwidth for their channels: \footcite[Cf.][481-482]{Pahlavan2009}

\begin{itemize}
  \item Channel 0: Located at 868.3 MHz with 1 MHz channel bandwidth
  \item Channels 1 to 10: Encompass the frequency band from 902 MHz to 928 MHz using a channel bandwidth of 2 MHz
  \item Channels 11 to 26: Starting from 2.4 GHz and ranging to 2.4835 GHz with a channel bandwidth of 5 MHz
\end{itemize}

The link layer has a built-in 128 bit \gls{aes} encryption and allows addressing of the nodes in unicast or broadcast communications. The physical payload length is 127 bytes. After the link layer framing 72 - 116 bytes of payload are available for the upper layers. \footcite[Cf.][18]{Shelby2009}

The IEEE 802.15.4 standard defines four different frame types: beacon, \gls{mac}, acknowledgement and data frames. \footcite[Cf.][]{RFC4944}

\begin{figure}[ht]
  \centering
  \includegraphics[
    width=12cm,
  ]{images/lowpan_packet_format}
  \caption{The IEEE 802.15.4 packet frame format, source: \cite[483]{Pahlavan2009}}
  \label{fig:lowpan_packet_format}
\end{figure}

\Cref{fig:lowpan_packet_format} shows all four types of IEEE 802.15.4 frames. Each frame starts with a 4 byte preamble, a 1 byte packet delimiter and 1 byte for start-of-the-frame. Each frame consists of a 2 byte frame control and 1 byte sequence number fields. All frames except the acknowledgement frame have an address field and a variable payload field followed by a \gls{crc}. \footcite[Cf.][]{RFC4944}

\section{ZigBee}
\label{sec:zigbee}

ZigBee is one of the major technologies based on the IEEE 802.15.4 standard. The major targets of ZigBee are to minimize the parameters power consumptation, data rate and duty cycle. ZigBee supports a set of channel rates ranging from 20 kbps to 250 kbps. \footcite[Cf.][545]{Kurose2012}

\subsection{Networking Architecture}

\begin{figure}[ht]
  \centering
  \begin{subfigure}[t]{4cm}
    \includegraphics[
      width=\textwidth,
    ]{images/zigbee_peer}
    \caption{Peer-to-Peer}
    \label{fig:zigbee_peer}
  \end{subfigure}
  ~
  \begin{subfigure}[t]{4cm}
    \includegraphics[
      width=\textwidth,
    ]{images/zigbee_star}
    \caption{Star}
    \label{fig:zigbee_star}
  \end{subfigure}
  ~
  \begin{subfigure}[t]{6cm}
    \includegraphics[
      width=\textwidth,
    ]{images/zigbee_cluster}
    \caption{Clustered Star}
    \label{fig:zigbee_cluster}
  \end{subfigure}
  \caption{The ZigBee network topologies, source: \cite[690]{Garg2007}}
  \label{fig:zigbee_topology}
\end{figure}

ZigBee defines two types of nodes: \glspl{ffd} and \glspl{rfd} nodes. A ZigBee network requires a least one \gls{ffd} node operating as a \gls{pan} coordinator. The \gls{ffd} can operate as a \gls{pan} coordinator, a coordinator or device. \glspl{ffd} can communicate with every other ZigBee node, so routing of messages through a ZigBee network could be implemented very easily. The \gls{rfd} nodes are very simple in functionality and resource allocation. They can only be used as a slave node to communicate with a \gls{ffd}. \footcite[Cf.][477-478]{Pahlavan2009}

ZigBee defines three topology types: Peer-to-Peer, Star and Clustered Stars. \Cref{fig:zigbee_peer} shows the Peer-to-Peer topology. This topology consists of a \gls{pan} coordinatorm \gls{ffd} coordinator nodes and one or more \gls{rfd} nodes. The coordinator nodes interconnect with each other and build a mesh network. This mesh network allows routing of messages through very efficiently. The Star topology, as shown in \cref{fig:zigbee_star}, is a master-slave architecture similar to the architecture of piconets in Bluetooth. The third topology, illustrated in \cref{fig:zigbee_cluster}, is the Clustered Star. \footcite[Cf.][689-691]{Garg2007}

\subsection{ZigBee Protocol Stack}

\begin{figure}[ht]
  \centering
  \includegraphics[
    width=7cm,
  ]{images/zigbee_protocol_stack}
  \caption{The ZigBee protocol stack, source: \cite[96]{Gomez2010}}
  \label{fig:zigbee_protocol_stack}
\end{figure}

The physical and the \gls{mac} layers are defined in the IEEE 802.15.4 standard and not captured by the ZigBee specification. The \gls{zdo} is part of every ZigBee node and its application end point address is always zero. The developers can define up to 240 custom application end points ranging from 1 - 240. Those application end points sit on top of the \gls{aps}, which routes messages from the lower layers to the correct application end point. The network layer defines control, alerting and monitoring functionality supporting the communication into the different network topologies. The security service defines authentication, encryption and key exchange functions for the network and application support layer. \footcite[Cf.][479-480]{Pahlavan2009}

\section{6LoWPAN}
\label{sec:6lowpan}

\gls{6lowpan} is defined by an IETF specification, the \gls{rfc} 4944. It is based on the IEEE 802.15.4 physical and link layers. The main difference to the previously discussed protocol standards is that \gls{6lowpan} is an \gls{ipv6} built on top of the IEEE 802.15.4 standard. \footcite[Cf.][]{RFC4944}

\subsection{Networking Architecture}

\begin{figure}[ht]
  \centering
  \includegraphics[
    width=12cm,
  ]{images/6lowpan_architecture}
  \caption{The \gls{6lowpan} architecture, source: \cite[14]{Shelby2009}}
  \label{fig:6lowpan_architecture}
\end{figure}

\Cref{fig:6lowpan_architecture} shows the different types of \glspl{lowpan}, which are stub networks for \glspl{6lowpan}: Simple \glspl{lowpan}, Extended \glspl{lowpan} and Ad-Hoc \glspl{lowpan}. Each \gls{lowpan} is a container for \gls{6lowpan} nodes. The Ad-Hoc \gls{lowpan} operates without infrastructure or Internet access. A Simple \gls{lowpan} has Internet connectivity through one \gls{lowpan} edge router, whereas the Extended \gls{lowpan} consists of multiple \gls{lowpan} edge router nodes. \footcite[Cf.][13]{Shelby2009}

\subsection{6LoWPAN Protocol Stack}

\begin{figure}[ht]
  \centering
  \includegraphics[
    width=4cm,
  ]{images/6lowpan_protocol_stack}
  \caption{The 6LoWPAN protocol stack, source: \cite[16]{Shelby2009}}
  \label{fig:6lowpan_protocol_stack}
\end{figure}

The \gls{6lowpan} protocol stack, as shown in \cref{fig:6lowpan_protocol_stack}, is nearly identical to a standard \gls{ipv6} stack. One of the differences is that \gls{6lowpan} only supports the \gls{ipv6} protocol. On top of the \gls{lowpan} an adaption layer has been introduced to support the \gls{ipv6} protocol. \gls{6lowpan} only provide packet based data transmission. Therefore, only the \gls{udp} and \gls{icmp} are supported, and no \gls{tcp} support is available, which is the next major difference to normal \gls{ipv6} protocol. \footcite[Cf.][16]{Shelby2009}

\section{Wi-Fi}
\label{sec:wifi}

One might think that this chapter should not exists because Wi-Fi defines any technology based on the IEEE 802.11 standards. One might be right, but the IEEE started a task group in 2009 with the aim to develop a standard for the 60 GHz band. Their main target was to develop a bridging technology between \gls{wlan} and \gls{wpan}. They formalized the IEEE 802.11ad standard. IEEE 802.11ad was built on top of the existing IEEE 802.11 standards and operates in the 2.4 GHz, the 5 GHz and the 60 GHz bands. The backwards compatibility to prior IEEE 802.11 standards is guaranteed due to the multiple physical layer technique, which allows the parallel operating of different IEEE 802.11 physical layers. \footcite[Cf.][1]{Zhu2011}

\subsection{Directional Communication}

\begin{figure}[ht]
  \centering
  \begin{subfigure}[t]{10cm}
    \includegraphics[
    width=\textwidth,
    ]{images/wifi_beamforming}
    \caption{Beamforming}
    \label{fig:wifi_beamforming}
  \end{subfigure}
  ~
  \begin{subfigure}[t]{5cm}
    \includegraphics[
    width=\textwidth,
    ]{images/wifi_sectors}
    \caption{Optimal Sector}
    \label{fig:wifi_sectors}
  \end{subfigure}
  \caption{The IEEE 802.11ad antenna sectors, source: \cite[2-7]{Nitsche2014}}
  \label{fig:wifi_directional}
\end{figure}

Due to the fact that 60 GHz signals suffer from increased attenuation depending on the distance, IEEE 802.11ad standard specifies a unique transmission scheme. The antenna azimuth is therefore intersected to form sectors. IEEE 802.11ad depict two possible implementation strategies: precomputed phased antenna arrays or multiple directional antennas. For excellent signal quality the peers have to use the optimal antenna sectors, as illustrated in \cref{fig:wifi_sectors}. The process of arranging the optimal sector pairs is called beamforming, as shown in \cref{fig:wifi_beamforming}. \footcite[Cf.][1-2]{Nitsche2014}

\subsection{Physical Layer}

IEEE 802.11ad defines three physical layers: The Control, Single Carrier and \gls{ofdm} layers. The packet structure is the same for all the layers. The Control layer is used to transmit beacons and sector management frames. The Single Carrier layer offers energy efficient data transmission at lower data rate of 385 Mbps. The maximum data rate can be achieved using the \gls{ofdm} layer. This optional layer can transmit data at rates up to 6.75 Gbps. \footcite[Cf.][3]{Nitsche2014}

\subsection{Networking Architecture}

\begin{figure}[ht]
  \centering
  \includegraphics[
    width=8.5cm,
  ]{images/wifi_pbss}
  \caption{IEEE 802.11ad \gls{pbss}, source: \cite{drcnet}}
  \label{fig:wifi_pbss}
\end{figure}

The \gls{pbss} is a new network type formalized in the IEEE 802.11ad standard. As shown in \cref{fig:wifi_pbss}, \gls{pbss} stations (STA) communicate in a peer-to-peer manner, where one STA have to take the role of a \gls{pbss} Control Point (PCP). Like an access point, the PCP announces the network and manages the medium access. \footcite[Cf.][4]{Nitsche2014}

\subsection{Packet Frame Formats}

\begin{figure}[ht]
  \centering
  \includegraphics[
    width=9cm,
  ]{images/wifi_packet_format}
  \caption{IEEE 802.11ad packet frame format, source: \cite[4]{Nitsche2014}}
  \label{fig:wifi_packet_format}
\end{figure}

In \cref{fig:wifi_packet_format} the IEEE 802.11ad packet frame format is shown. It consists of a short training field (STF), a channel estimation field (CEF), followed by the physical header (PHY Header). The \gls{mac} header, the \gls{mac} payload and the \gls{crc} fields, form the physical payload (PHY Payload). The last fields, automatic gain control (AGC) and training (TRN) are unique for the IEEE 802.11ad beamforming technique. \footcite[Cf.][3]{Nitsche2014}

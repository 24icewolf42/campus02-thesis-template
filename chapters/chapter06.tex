\chapter{Evaluation and Comparison}
\label{ch:eval}

\begin{table}[h!]
  \renewcommand{\arraystretch}{2}
  \newcolumntype{P}[1]{>{\raggedright}p{#1}}
  \centering
  \begin{tabular}{ P{2.3cm} P{1.8cm} P{1.8cm} P{1.8cm} P{2cm} P{2cm} >{\arraybackslash}P{1.8cm} }
    \toprule
    \rowcolor{Gainsboro}
    \textbf{Parameters} & \textbf{Bluetooth} & \textbf{Bluetooth Smart} & \textbf{UWB} & \textbf{ZigBee} & \textbf{6LoWPAN} & \textbf{Wi-Fi} 802.11ad \\
    \midrule
    \textbf{Chipset} & CC2560 \footnotemark & CC2640 \footnotemark & {PL3120 \footnotemark \\ PL3130 \footnotemark} & CC2630 \footnotemark & CC2630 \footnotemark & Wil6200 \footnotemark \\
    \textbf{Frequency Band} & 2.4~GHz to 2.48~GHz & 2.4~GHz to 2.48~GHz & 3.1~GHz to 5.8~GHz & 868.3~MHz / 902~MHz to 928~MHz / 2.4~GHz to 2.48~GHz & 868.3~MHz / 902~MHz to 928~MHz / 2.4~GHz to 2.48~GHz & 57.24~GHz to 65.88~GHz \\
    \textbf{Number of Channels} & 79 & 40 & 3 & 27 & 27 & 4 \\
    \textbf{Channel Bandwidth} & 1~MHz & 2~MHz & 528~MHz & 1~MHz / 2~MHz / 5~MHz & 1~MHz / 2~MHz / 5~MHz & 2.16~GHz \\
    \textbf{Modulation} & \gls{gfsk} & \gls{gfsk} & \gls{ofdm} & \gls{dsss} & \gls{dsss} & \gls{ofdm} \\
    \textbf{Bandwidth Spreading} & \gls{fhss} & \gls{fhss} & \gls{dsss} & \gls{dsss} & \gls{dsss} & \gls{ofdm} \\
    \textbf{Max. Data Rate} & 2.1~Mbps & 0.27~Mbps & 675~Mbps & 20~kbpss / 40~kbps / 250~kbps & 20~kbps / 40~kbps / 250~kbps & 6.75~Gbps \\
    \textbf{Transmission Range} & 10~m & 100~m & 10~m & 10~m & 10~m & 10~m \\
    \textbf{Energy Consumption} & 1~W & 500~mW & 330~mW & 20~mW & 20~mW & 700~mW \\
    \textbf{Network Type} & Piconet / Scatternet & Scatternet & Piconet & Scatternet & Piconet & Peer-to-Peer \\
    \textbf{Max. Number of Nodes} & 7 & 5917 & 16 & 1024 & 100 & 1 \\
    \textbf{Latency} & 100~ms& 6~ms & <1~ms & 100~ms & 100~ms & 10~{\textmu s} \\
    \textbf{Mobility} & Yes & Yes & Yes & Yes & Yes & No \\
    \bottomrule
  \end{tabular}
  \caption{Parameters of \gls{wpan} technologies, source: Own diagram}
  \label{table:parameters}
  \vspace{-1.5em}
\end{table}
\footcitetext[Cf.]{cc2560}
\footcitetext[Cf.]{cc2640}
\footcitetext[Cf.]{pl3120}
\footcitetext[Cf.]{pl3130}
\footcitetext[Cf.]{cc2630}
\footcitetext[Cf.]{cc2630}
\footcitetext[Cf.]{wil6200}

The parameters of the wireless networking technologies, shown in \cref{table:parameters}, were researched from chipset manufacturers and combined with the information of \cref{ch:wpan}. Furthermore, the following comparison can be conducted as illustrated in \cref{table:comparison}.

The applicability of the Bluetooth technology is far away from the expectations. Because of the low data rate, Bluetooth is not usable for \gls{when} applications. The reason for the ordinary rating in the other \gls{whn} categories, is due to the small number of network nodes.

A totally different picture shows Bluetooth Smart. Caused to the low data rate, the Bluetooth Smart technology is not well suitable for \gls{when} applications. But Bluetooth Smart is a good choice for \gls{whan} and wireless \gls{m2m} networks. Both network types have the requirements of very low power consumption, a scalable network architecture and the support for hundreds a network nodes. Bluetooth Smart offer a good coverage range and is also applicable for outdoor sensors. But Bluetooth Smart is not very easily integratable into heterogenous \gls{whn} environments, due to the non-\gls{ip} stack.

\begin{table}[h!]
  \renewcommand{\arraystretch}{2}
  \newcolumntype{P}[1]{>{\raggedright}p{#1}}
  \centering
  \begin{tabular}{ P{3cm} P{2.5cm} P{2.5cm} P{2.5cm} >{\arraybackslash}P{2.5cm} }
    \toprule
    \rowcolor{Gainsboro}
    \textbf{Wireless Networking Technologies} & \textbf{Wireless Home Entertainment Network} & \textbf{Wireless Home Automation Network} & \textbf{Wireless \gls{m2m} Network} & \textbf{Wireless Home Network} \\
    \midrule
    \textbf{Bluetooth} & Low & Medium & Medium & Medium \\
    \textbf{Bluetooth Smart} & Low & High & High & Medium \\
    \textbf{UWB} & High & Medium & Medium & Low \\
    \textbf{ZigBee} & Low & High & High & Medium \\
    \textbf{6LoWPAN} & Low & High & High & High \\
    \textbf{Wi-Fi 802.11ad} & High & Low & Low & Medium \\
    \bottomrule
  \end{tabular}
  \caption{Aptitude of \gls{wpan} technologies in Smart Home networks, source: Own diagram}
  \label{table:comparison}
  \vspace{-1.5em}
\end{table}

The \gls{uwb} technology is very well suited for \gls{when}. \gls{uwb} offer a high data rate of several hundreds of Mbps. The technology is very robust to interference and has low power consumption. \gls{uwb} is a versatile technology, just the number of network nodes is not large enough to be the first choice in \gls{whan} and wireless \gls{m2m} networks. The non-\gls{ip} stack is also a barrier for heterogenous \gls{whn} environments.

ZigBee is the technology for the \gls{iot}, therefore the rating is high in the respective categories, \gls{whan} and wireless \gls{m2m} networks. Low power consumption and a large number of network nodes are the main requirements for those network types. ZigBee does not offer high data rates, nor is the protocol not \gls{ip} conform. Therefore, ZigBee is not a good choice for \gls{when} or heterogenous \gls{whn} environments.

The next candidate makes that possible, \gls{6lowpan}: Easy integratable into \gls{ip} networks, low power consumption and a high maximum number of network nodes. \gls{6lowpan} is well suited for \gls{whan}, wireless \gls{m2m} networks and heterogenous \gls{whn} environments. Because of the low data rate, \gls{6lowpan} is not very attractive to \gls{when}.

The escapee from the Wi-Fi technologies, the IEEE 802.11ad standard, is very well suited for high speed data transmissions as needed in \glspl{when}. But, the number of simultaneous data transfers to different network nodes is very limited. That is why the IEEE 802.11ad is not applicable in \gls{whan} or wireless \gls{m2m} networks. Due to the fact that IEEE 802.11ad \gls{mac} layer is combinable with other IEEE 802.11 standards in multi-\gls{mac} chipsets, the IEEE 802.11ad is integratable in heterogenous \gls{whn} environments.

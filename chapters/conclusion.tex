\chapter{Conclusion and Future Work}
\label{ch:conclusion}

Due to the wide spread of wireless networking technologies along most domains of our life, more and more attention has been focused on ecological parameters of these technologies. The main challenge from the many difficulties involved in the selection of wireless networking technologies is in providing a guidance to the aptitude for a specific domain. It is necessary for a large number of wireless applications to provide an energy efficient overall picture, because power consumption is a crucial parameter for decisions on selecting wireless products and services.

The sector of ecologically designed Smart Homes is part of the concentrated research on energy efficient wireless networking technologies. Because of the many and various wireless networking types in Smart Homes, the requirements for applications and services in Smart Homes spread over multiple parameters.

This thesis presents a guidance for the selection of wireless networking technologies in Smart Homes based on their standards and implementations by various manufacturers. The first part of this thesis introduces the different wireless network types according to their coverage range and gives an overview to the wireless Smart Home networks. Examples are given, taken from their respective area of application and the key issues of wireless Smart Home networks are introduced.

The subject of the second part of this thesis is the introduction of probably suitable wireless networking technologies for energy efficient Smart Home applications. In order to answer the research question I reviewed six energy efficient wireless networking technologies used in \glspl{pan}. Those technologies are all open standards, and were developed and maintained by different alliances.

The discussed \gls{wpan} technologies can be found in Smart Home applications in every corner and their possibilities are vast. All of these technologies are promising within their domain but Bluetooth will be less significant. Especially the network topology is a downside of Bluetooth. \gls{uwb} and IEEE 802.11ad offer extreme high data rates, and ZigBee and \gls{6lowpan} are both technologies for larger sensor and actuator networks.

The comparison of the \gls{wpan} technologies in conjunction with their aptitude for different Smart Home applications shows that every technology has its hotspots and right to exist in the Smart Home networking environment.

It would be interesting to investigate the aptitude of the \gls{wpan} technologies with experiments in simulated environments. Such simulations could deliver accurate results, which completes this guidance by practical experience with the focus on energy-efficiency, security and coexistence. These parameters are meaningful areas for further study, to adapt and develop new Smart Home products.

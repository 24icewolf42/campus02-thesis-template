%% Scientific Article / Thesis template
%%
%% Author: Andreas Landgraf (2015)
%%         andreas dot landgraf at edu dot campus02 dot at
%%
%%%%%%%%%%%%%%%%%%%%%%%%%%%%%%%%%%%%%%%%%%%%%%%%%%%%%%%%%%%%%

\documentclass[
	english,    % select document language: english or german
	thesis,    % choose document type: thesis or article
]{scientific}

% ~~~~~~~~~~~~~~~~~~~~~~~~~~~~~~~~~~~~~~~~~~~~~~~~~~~~~~~~~~~~~~~~~~~~~~~~
% Load bibliography %%%%%%%%%%%%%%%%%%%%%%%%%%%%%%%%%%%%%%%%%%%%%%%%%%%%%%
% ~~~~~~~~~~~~~~~~~~~~~~~~~~~~~~~~~~~~~~~~~~~~~~~~~~~~~~~~~~~~~~~~~~~~~~~~

\addbibfile{literature/mendeley.bib}
\addbibfile{literature/internet.bib}

%%%----------------------------------------------------------
%%% Glossary/Acronym list
%%%----------------------------------------------------------
\newacronym{MFD}{MFD}{mode field diameter}
\newacronym{CPA}{CPA}{chirped pulse amplification}
\newacronym{NA}{NA}{numerical apertur}
\newacronym{MMI}{MMI}{multi-mode interference}
\newacronym{SLM}{SLM}{spatial light modulator}
\newacronym{LCD}{LCD}{liquid crystal display}
\newacronym{px}{px}{Pixel}
\newacronym{DNA}{DNA}{deoxyribonucleic acid}
\newacronym{DOF}{DOF}{depth of focus}
\newacronym{PSF}{PSF}{point spread function}
\newacronym{SNOM}{SNOM}{scanning nearfield optical microscope}
\newacronym{FWHM}{FWHM}{full width at half maximum}
\makeglossaries
%%%----------------------------------------------------------

% ~~~~~~~~~~~~~~~~~~~~~~~~~~~~~~~~~~~~~~~~~~~~~~~~~~~~~~~~~~~~~~~~~~~~~~~~
% Document %%%%%%%%%%%%%%%%%%%%%%%%%%%%%%%%%%%%%%%%%%%%%%%%%%%%%%%%%%%%%%%
% ~~~~~~~~~~~~~~~~~~~~~~~~~~~~~~~~~~~~~~~~~~~~~~~~~~~~~~~~~~~~~~~~~~~~~~~~

% Information about the document (used for building the titlepage)
\subject{Thesis}
\title{\LaTeX{} Scientific Document Template}
\author{Andreas Landgraf}
\city{Graz}
\date{13}{11}{2015}
\supervisor{Dr. techn. Hans Dieter Toska}
\secondreviewer{Dr. ir. Thomas Karl Wurst, Associate Professor}
\institute{Bachelor Degree Programme \par Innovation Management}
\university{Fachhochschule der Wirtschaft}
\logo{images/Campus02_Logo}

%%% document start %%%%%%%%%%%%%%%%%%%%%%%%%%%%%%%%%%%%%%%%%%%%%%%%%%%%%%%%%%
\begin{document}
%%%%%%%%%%%%%%%%%%%%%%%%%%%%%%%%%%%%%%%%%%%%%%%%%%%%%%%%%%%%%%%%%%%%%%%%%%%%%

% Configure page numbering - required for hyperref (not displayed)
\pagenumbering{alph}\setcounter{page}{1}
\pagestyle{empty}

%%%----------------------------------------------------------
% -- title page --
\maketitle
% -- abstract --
\makeabstract{
  This work deals with the introduction of the subject 'Wissenschaftliches Arbeiten' at ...
}{
  Die vorliegende Arbeit befasst sich mit der Einführung des Kurses 'Wissenschaftliches Arbeiten' an der ...
}
% -- declaration --
\makedeclaration
%%%----------------------------------------------------------

%%%----------------------------------------------------------
\pagestyle{scrheadings}
\pdfbookmark[1]{\contentsname}{toc}
\tableofcontents
\clearpage
\printglossary[type=\acronymtype]
\printglossary[style=altlist]
%%%----------------------------------------------------------

\chapter{Introduction}

This \LaTeX{} template is designed for the creation of thesis documents (bachelor, master, phd) and targets both beginner and experienced users of \LaTeX{}. It supports all basic functionality and requirements of a technical document such as the inclusion of graphics, math, tables, references, bibliography and much more. In contrast to a standard LaTeX document this template not only loads all state of the art packages (\path{scientific.cls}) to provide the best functions for each task, but also includes a separate document for the style/layout of the document (\path{style.tex}). It therefore tries to separate functionallity and layout as much as possible. And the best, everything is documented in the code.

\chapter{Theory}

Duis porta orci. Integer eu arcu at enim tempus facilisis. Pellentesque dignissim orci sed est. Etiam elementum laoreet mi. Donec nunc sapien, dictum in, tristique sed, aliquam vitae, massa. Morbi magna magna, vestibulum tempor, lobortis non, convallis nec, nibh. In sed nibh. Suspendisse adipiscing dictum pede. Suspendisse non augue. Lorem ipsum dolor sit amet, consectetuer adipiscing elit. Pellentesque lacinia, velit sed commodo convallis, diam dolor consequat ligula, a scelerisque quam neque et purus. Praesent vel augue. Sed lectus leo, dignissim eget, vulputate eu, auctor ut, nulla. Vivamus a quam. Nulla tellus. Pellentesque tempor pulvinar nunc.

\section{Section heading}

Fusce vitae quam eu lacus pulvinar vulputate. Suspendisse potenti. Aliquam imperdiet ornare nibh. Cras molestie tortor non erat. Donec dapibus diam sed mauris laoreet volutpat. 

\subsection{Subsection heading}

Aliquam dignissim laoreet mi. Duis pulvinar nulla id velit euismod fringilla. In hac habitasse platea dictumst. Vestibulum dolor tellus, gravida a, condimentum nec, laoreet ut, nisi. Aenean lacus purus, tristique in, sagittis sit amet, pellentesque non, neque. Sed egestas nibh vitae velit. Nunc adipiscing. Donec sed lectus. Donec ultrices lacus nec orci. Fusce sit amet nulla. Suspendisse vulputate, mi nec nonummy sodales, ligula massa molestie est, at sagittis nisi est in leo.

\subsubsection{Subsubsection heading}

Mauris rutrum volutpat massa. Suspendisse potenti. Nam varius. Fusce nec leo. Morbi vestibulum augue ac justo. Vivamus in odio in turpis pharetra blandit. Mauris aliquet ullamcorper libero. Integer quam mi, venenatis ut, tristique ut, tempus at, ipsum. Donec malesuada. In quis tellus et ipsum hendrerit imperdiet. Vivamus sapien ipsum, suscipit sed, gravida a, lacinia laoreet, ipsum. Quisque augue. Nulla justo enim, auctor at, hendrerit nec, porttitor non, urna. Duis tincidunt tincidunt dui. Lorem ipsum dolor sit amet, consectetuer adipiscing elit. Suspendisse potenti. Aenean sit amet mauris. In luctus purus nec lorem. Proin orci tortor, tempus sit amet, molestie hendrerit, placerat egestas, tellus.

\chapter{Experiments}

\gls{MFD}
\gls{CPA}
\gls{NA}
\gls{MMI}
\gls{SLM}
\gls{LCD}
\gls{px}
\gls{DNA}
\gls{DOF}
\gls{PSF}
\gls{SNOM}
\gls{FWHM}

\chapter{Results}

\begin{figure}[ht]
	\centering
	\missingfigure[figwidth=6cm]{Testing a long text string}
	\caption{Typical figure, source: Own diagram}
	\label{fig:1}
\end{figure}

\chapter{Summery and Outlook}

\begin{table}[h!]
	\centering
	\begin{tabular}{||c c c c||}
    \hline
    Col1 & Col2 & Col2 & Col3 \\ [0.5ex] 
    \hline\hline
    1 & 6 & 87837 & 787 \\ 
    2 & 7 & 78 & 5415 \\
    3 & 545 & 778 & 7507 \\
    4 & 545 & 18744 & 7560 \\
    5 & 88 & 788 & 6344 \\ [1ex]
    \hline
    \end{tabular}
	\caption{Typical table example}
	\label{table:1}
\end{table}

%%%----------------------------------------------------------
\nocite{*}
\makebibliography
\listoffigures
\listoftables
%%%----------------------------------------------------------

\end{document}